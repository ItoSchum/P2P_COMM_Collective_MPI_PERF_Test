% Experiment results

For this experiment, the message size chanegs from 1 to 1024 by the power of two.
And the rank number per node changes from 1 to 6 for up to two nodes.
After the testing in different conditioins, it is found that the lantency of the network is almost static which is fixed around 0.004 ms.

\begin{figure*}[h]
\centering
\hspace*{\fill}
\subfloat[Rank 1]{
  \includegraphics[width=0.45\linewidth]{exe_time_vs_messsage_size_rank_1.png}
}
\hspace*{\fill}
\subfloat[Rank 2]{
  \includegraphics[width=0.45\linewidth]{exe_time_vs_messsage_size_rank_2.png}
}
\hspace{0mm}
\subfloat[Rank 3]{
  \includegraphics[width=0.4\linewidth]{exe_time_vs_messsage_size_rank_3.png}
}
\hspace*{\fill}
\subfloat[Rank 4]{
  \includegraphics[width=0.45\linewidth]{exe_time_vs_messsage_size_rank_4.png}
}
\hspace{0mm}
\subfloat[Rank 5]{
  \includegraphics[width=0.45\linewidth]{exe_time_vs_messsage_size_rank_5.png}
}
\hspace*{\fill}
\subfloat[Rank 6]{
  \includegraphics[width=0.45\linewidth]{exe_time_vs_messsage_size_rank_6.png}
}
\caption{Estimated execution time vs message size along different ranks.}
\end{figure*}


\begin{figure*}[h!]
\hspace{0mm}
\subfloat[Comparison of estimated execution time vs message size across different ranks.]{
	\includegraphics[width=0.5\linewidth]{estimated_execution_time_vs_message_size.png}	
}
\hspace*{\fill}
\subfloat[Estimated execution time in relation to ranks]{
	\includegraphics[width=0.5\linewidth]{estimated_execution_time_vs_number_of_ranks_1.png}
}
\caption{The relationship between message sizes and number of ranks to estimated execution time.}
\end{figure*}


\begin{figure}[h]
\includegraphics[width=0.5\textwidth]{estimated_execution_time_vs_number_of_ranks.png}
\caption{An alternative interpretation of the relations between estimated execution time and the number of ranks.}
\end{figure}