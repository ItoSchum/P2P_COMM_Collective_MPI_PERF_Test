% Game of Life

The Game of Life as invented by the British mathematician John Horton Conway in 1970 \cite{gardener1970mathematical} is a cellular automaton. The algorithm is a zero-player game and as the game evolves throughout undetermined number of iterations, the outcome is determined by the given initial configuration. The game is taken place on a two-dimensional orthogonal grid of square cells. The cell status is atomic, that is it can only be found as alive or dead. Each cell's status is determined by eight adjacent neighboring cells. At each step in time, any cell with fewer than two live neighbors dies due to under-population. On the contrary, if the cell lives with two or three live neighbors survives to the next step. If there are more than three immediately adjacent cells, the cell perishes due to over-population. Lastly, the cell resucitates with exactly three live neighbors and can be seen as the result of cellular reproduction. 

The operations involved in the Game of Life include instantiation of the world configuration, the update of the cells in the world, and swaping the newly updated world with the previous one. The operations not only is possible to implement serially, but also with parallel speed-up mechanism to take advantage of the efficient memory manipulation offered by NVIDIA CUDA math library, or through the de-facto networking of various nodes through MPI libraries when the dimension of the world increases toward dimension of high orders of magnitude. For the message passing optimization, we have to take care of the "ghost" rows at the MPI rank boundaries apart from allocating suitable memory chunk for the rank itself. To achieve such requirement, special care needs to be taken at sending and receiving messages at the boundaries where the current rank\'s top ghost row meets with the previous rank\'s last row, and where the bottom ghost row meets the first row of the next rank. In addition, the CUDA kernel also needs to undergo modifications to account for the lack of top and bottom edge \"world wrapping\". Therefore, at the initialization of the world configuration, the placement of elements at the edge of the world should be placed in determined ranks. Once the instantiation of the initial world configuration takes place, the MPI-and-CUDA-enabled Game of Life program opens way for further bencharking. 

