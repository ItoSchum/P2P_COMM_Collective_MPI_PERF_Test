% Performance Metrics

To estimate the execution time according to Hockney's model,
we used the formula \textbf{t(s) = l + s / b}. In the formula,
\textbf{s} stands for the message size,
\textbf{l} stands for the latency of the network,
\textbf{b} stands for the bandwidth of the network.

\begin{equation*}
    \begin{aligned}
    T = & \alpha + \beta \\
    \alpha = & l = Latency \\
    \beta = & \frac{s} {b} = \frac{Message Size} {Bandwidth}\\
    \end{aligned}
\end{equation*}

Then it came with the quesion "How can we determine the latency and the bandwidth?"
In this benchmark model, it assumes that some process A sends a message to process B, while process B sends message back
The advantage is that it will not require any synchronized clocks between these two processes.
While the disadvantage is that it will presume the communication performance or the costs between two points is totally symmetric.

To determine latency, it reuires to execute the benchmark for itration time equal to zero.

For this experiment, the message size chanegs from 1 to 1024 by the power of two.
And the rank number per node changes from 1 to 6 for up to two nodes.
After the testing in different conditioins, it is found that the lantency of the network is almost static.
More specifically, to calculate the total execution time,
it simply subtracts the end timestamp generated after the for loop of iteration
by the start timestamp, both of which are generated via the function \textbf{MPI\_Time()}.
